\documentclass[10pt,a4paper]{article}
\usepackage[utf8]{inputenc}
\usepackage{amsmath}
\usepackage{amsfonts}
\usepackage{amssymb}
\usepackage{graphicx}
\usepackage{listings}
\usepackage{color}
\definecolor{dkgreen}{rgb}{0,0.6,0}
\definecolor{gray}{rgb}{0.95,0.95,0.95}
\definecolor{darkgray}{rgb}{0.7,0.7,0.7}

\definecolor{mauve}{rgb}{0.58,0,0.82}

\lstset{frame=single,
  frameround=ffff,
  language=R,
  aboveskip=3mm,
  belowskip=3mm,
  showstringspaces=false,
  columns=flexible,
  basicstyle={\small\ttfamily},
  numbers=none,
  numberstyle=\tiny\color{gray},
  %keywordstyle=\color{blue},
  commentstyle=\color{dkgreen},
  stringstyle=\color{mauve},
  backgroundcolor=\color{gray},
  rulecolor=\color{darkgray},
  breaklines=true,
  breakatwhitespace=true,
  tabsize=3
}
\begin{document}
\section{Question 1}
\subsection{Design an R function (or use another programme if you prefer) to evaluate the
empirical ACF for a given series up to a specified maximum lag and plots it along
with a data plot.}


\begin{lstlisting}
my.acf <- function(y, lag) {
  
  N <- length(y)
  y.mean <- mean(y)
  
  out <- numeric(lag + 1)
  out[1] <- var(y)
  
  # Evaluating ACF for each lag
  for(h in 2:(lag + 1)) {
    temp <- 0.0
    for(t in 1:(N-(h-1))) {
      temp <- temp + (y[t] - y.mean)*(y[t+(h-1)] - y.mean)
    }
    out[h] <- temp/N
  }
  
  # Plotting ACF and time series
  par(mfrow = c(2, 1))
  grid <- seq(from = 0, to = lag, len = length(out))
  plot.ts(y, type = "l", ylab = expression(Y[t]))
  plot(grid, out, type = "h", main = "ACF", xlab = "Lag", ylab = "ACF")
  abline(h = 0)
  par(mfrow = c(1, 1))
  
}
\end{lstlisting}


\subsection{Take $Y_t$ = $\epsilon_t$ where $\epsilon_t$
is Gaussian white noise.}


\end{document}